% Created 2018-07-19 Thu 11:28
% Intended LaTeX compiler: pdflatex
\documentclass[11pt]{article}
\usepackage[utf8]{inputenc}
\usepackage[T1]{fontenc}
\usepackage{graphicx}
\usepackage{grffile}
\usepackage{longtable}
\usepackage{wrapfig}
\usepackage{rotating}
\usepackage[normalem]{ulem}
\usepackage{amsmath}
\usepackage{textcomp}
\usepackage{amssymb}
\usepackage{capt-of}
\usepackage{hyperref}
\author{Milo Cress}
\date{\textit{<2018-07-01 Sun>}}
\title{Project Description}
\hypersetup{
 pdfauthor={Milo Cress},
 pdftitle={Project Description},
 pdfkeywords={},
 pdfsubject={},
 pdfcreator={Emacs 26.1 (Org mode 9.1.6)}, 
 pdflang={English}}
\begin{document}

\maketitle
\tableofcontents


\section{{\bfseries\sffamily TODO} Problem}
\label{sec:org94bf369}
As computer graphics advance in processing power, the visual detail that graphics processors can render also increases, requiring a larger amount of data to render. The goal of this project is to explore on-the-fly creation of 3-dimensional rendering data, which would limit the amount of data required to render a scene by generating visually relevant areas at a higher level of detail than non-visually relevant areas. While traditional models are stored at a consistent level of detail throughout the scene, this project attempts to create models that comprise only a fraction of the scene, but relative to the viewer, appear to comprise the entire scene.
\section{{\bfseries\sffamily TODO} Solution}
\label{sec:orgcae93ad}
This project uses procedural terrain generation that functions on a variety of spatial and temporal scales, at a realistic level of detail. It is divided into two parts: the sector event buffer, and the Sector Tree.
\subsection{{\bfseries\sffamily TODO} Sector Event Buffer}
\label{sec:orgae3604d}
Implementation of procedural terrain generation with fixed geometries and Levels of Detail, as well as simulation of natural events such as erosion, continental drift, etc.

We will use the programming language, Haskell, to prototype our system:
\begin{verbatim}
main :: IO ()
main = putStrLn "test"
\end{verbatim}

\subsection{{\bfseries\sffamily TODO} Sector Tree}
\label{sec:org569f2ce}
Implementation of a \texttt{SectorTree}, along with code that divides sectors into smaller child sectors, as well as control code that decides how/when to expand or prune branches of the sector tree.
\end{document}
